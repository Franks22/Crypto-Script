\chapter{RSA}

The name \textsc{RSA}\index{RSA} comes from the initials of \textsc{Rivest}, \textsc{Shamir}, \textsc{Ademan}. In 1977 it was the first encryption and signature system with a public key.

\section{Generating the Keys}

We pick two large primes $p \neq q$, put $n=p \cdot q$ and compute $\varphi(n) = (p-1)(q-1)$.

Now pick an integer $e$, which is coprime with $\varphi(n)$.

Compute $e^{-1} \equiv d \bmod \varphi(n)$ using \textsc{XGCD}. \\

RSA public key: ($e,n$)
private key: d

\begin{example} \ \\
$p = 5$, $q = 7$, \\
$n = 5 \cdot 7 = 35$ \\
$\varphi(n) = (5-1)(7-1)=24$ \\
\begin{itemize}
\item Pick $e=5$, $gcd(5,24) = 1$, \\
$5^{-1} \equiv 5 \bmod 24$, so $d= 5$
\item Other options, e.g. $e = 7$. Compute $gcd$:
\[
\begin{matrix}
24 & 1 &0 & \\
7 & 0 & 1& q=3 \\
3 & 1 & -3 & q=2 \\
1 & -2 & 7& \\
\textcolor{ocre}r\textcolor{ocre}\uparrow \  & \textcolor{ocre}a\textcolor{ocre}\uparrow \  & \textcolor{ocre}b\textcolor{ocre}\uparrow \  & \\
\end{matrix}
\]
For every row, $q$ is the number of times the first item in the row ``fits into'' the item above it. So for $24$ and $e=7$ this means: $q= 34$ (since $7 \cdot 3 = 21$ with rest $3$). You then subtract 3 times this row from the previous row, or more general: subtract q times row from previous.

Every row now satisfies that $r = a \cdot 24 + b\ \cdot 7$, e.g. $1 = -2 \cdot 24 + 7 \cdot 7 \Rightarrow 7^{-1} \equiv 7 \bmod 24$
\end{itemize}
Then ($5,35$) public key and $d=5$ secret key or
($7,35$) public key and $d=7$ secret key.
\begin{remark}
It is pure coincidence (and due to the fact that we use small numbers) that $d$ and $e$ are equivalent)
\end{remark}
\end{example}

\section{Encryption and Decryption of message}

\subsection{Encryption}
We want to encrypt the message $m < n, m \in \N$.
\[
c \equiv m^e \bmod n
\]
\begin{remark}
At this point in an implementation we would use the Square and Multiply method.
\end{remark}

\subsection{Decryption}
To decrypt of c, we calculate:
\[
m' \equiv c^d \bmod n
\]
\begin{remark}
We now want to proof that the decryption undos the encryption, or as we stated earlier $D(E(m)) = m$, or in this case $m' = m$
\end{remark}
This decryption works because
\[
m' \equiv c^d \equiv (m^e)^d \equiv m^{e \cdot d} \equiv m^{1+k \varphi(n)} \equiv m \bmod n
\]

where $e \cdot d = 1 \bmod \varphi(n)$, i.e. $e \cdot d = 1 + k \cdot \varphi(n)$ for some $k$. 

For the last step, we used \textsc{Lagrange} with group $(\nicefrac{\Z}{n})^{\ast}$ because there $m^{\varphi(n)} \equiv 1 \bmod n$. You can check that $m^{1+k \varphi(n)} \equiv m \bmod n$ even when $gcd(m,n) \neq 1$. \\

To show this, use the \textsc{Chinese Remainder Theorem (CRT)}\index{Chinese Remainder Theorem}
\[ m \equiv 0 \bmod p \\
m \equiv a \bmod q, \,\, a \neq 0, \text{ then} \\
m^{e \cdot d} \equiv 0^{e \cdot d} \equiv 0 \bmod p
\]

In $(\nicefrac{\Z}{q})^{\ast}$ we have $a^{q-1} \equiv 1 \bmod q$.

Thus

\begin{align}
	m^{e \cdot d} \equiv a^{e \cdot d} &\equiv a^{1+k \varphi(n)} \equiv a^{1 + (p-1)(q-1)} \equiv a \cdot \big( \underbrace{a^{q-1}}_{=1}\big)^{p-1} \equiv a \cdot 1^{p-1} \equiv a \bmod q  \notag \\
	m^{e \cdot d} &\equiv 0 \equiv m \bmod p \notag \\
	m^{e \cdot d} &\equiv a \equiv m \bmod q, \notag
\end{align}

thus \textsc{CRT} gives $m^{e \cdot d} \equiv m \bmod n$.

Last case $0^{e \cdot d} \equiv 0 \bmod d \Rightarrow \text{ \textsc{RSA} gives } m' = m$.

\section{Making the Algorithm Faster}

To get a faster encryption, we can use a small $e$ when generating the key.

Choosing small $d$ gives \textsc{Eve} your key; but we can decrypt faster using \textsc{RSA-CRT}: Compute
\[
c^d \equiv m_p \bmod p \\
c^d \equiv m_q \bmod q \\
\]

This is faster because the operands are smaller (naivley this save a factor of $4$ in each computation, do this twice, so save factor of $2$).

Combine $m_p$ and $m_q$ using \textsc{CRT} to get $m$. \\

We can save more effort by noticing that $c^{p-1} \equiv \bmod p$, so compute $d_p \equiv d \bmod p-1$ and $d_q \equiv d \bmod q-1$; $d_p$ and $d_q$ have half the bit length of $d$, so the \textsc{Square and Multiply} loop is half as long in computing
\[
c^{d_p} \equiv m_p \bmod p \\
c^{d_q} \equiv m_q \bmod q \\
\]

Combine $m_p$ and $m_q$ to $m$ using \textsc{CRT}. This needs just one multiplication given $u \equiv p^{-1} \bmod q$ as precomputed value.

Then $m = m_p q \cdot v + p \cdot u \cdot m_q$ with $v \equiv q^{-1} \bmod p$. ``Naively'' because with \textsc{Karatsuba}, doublelength costs only 3 times as much and for very long integers the \textsc{Fast Fourier Transformation (FFT)}\index{Fast Fourier Transformation} takes only twice as long.