\chapter{Diffie-Hellman Key Exchange}\index{Diffie-Hellman Key Exchange}

\section{Introduction}

The \textsc{Diffie-Hellman Key Exchange} is used to secretly exchange keys for an encryption. The key exchange can be explained easily with an example:

\textsc{A(lice)} and \textsc{(B)ob} want to exchange a key, without letting the eavesdropper \textsc{E(ve)} know. Everybody (including) \textsc{Eve} knows a cyclic group $G$, a generator $g$ and its order $n$:

\begin{figure}[H]
  \centering\import{Chapter5/Pictures/}{DH.pdf_tex}
  \caption{Illustration of the\textsc{Diffie-Hellman key exchange}}{\textsc{A} and \textsc{B} both think of a number between $0$ and $n$ (this is $a$ and $b$ respectively). Now the compute $g^a$ and $g^b$ and exchange this information (so \textsc{Eve} knows $g^a$ and $g^b$. Then \textsc{A} computes $g^{b^a}$ and \textsc{B} computes $g^{a^b}$. So they both now know $g^{ab}$.}
  \label{fig:caesar_cipher}
\end{figure}

In summary, this means that \textsc{A} computes $h_b^a = (g^b)^a = g^{ba} = g^{ab}$ and \textsc{B} computes $h_a^b = (g^a)^b = g^{ab}$. So \textsc{A} and \textsc{B} both share $g^{ab}$, while Eve only knows $g$, $h_a = g^a$, $h_b =g^b$.

\section{Computational DH Problem (CDHP)}\index{Computational DH Problem (CDHP)}

Given $g$, $g^a$, $g^b$ compute $g^{ab}$.

In groups that are good for crypto there are no efficient attacks on the \textsc{CDHP}.

\begin{example} Examples for these ``good groups'' are:
\begin{itemize}
\item elliptic curves over finite fields
\item multipl. groups of finite fields
\end{itemize}
\end{example}

\section{Decisional DH Problem (DDHP)}\index{Decisional DH Problem (DDHP)}

Given $g$, $g^a$, $g^b$ and $g^c$.

Decide whether $g^c = g^{ab}$.

Proofs for protocols often use \textsc{DDHP} rathen than \textsc{CDHP}.

Bad groups would be: $\langle 2 \rangle \subseteq \Q^{\ast}$:
\[
1,2,4,18,16 \cdots, 2^i, \dots  \qquad \text{no reduction}
\]
$h_a=2^a$, \textsc{E} takes $\log$ and gets $a$.

\section{Discrete Logarithm Problem (DLP)}\index{Discrete Logarithm Problem}

Given $g$ and $g^a$, compute $a$.

Solving \textsc{DLP} implies solving \textsc{CDHP} implies solving \textsc{DDHP}. Usually \textsc{DLP} is the best attack we know on \textsc{DHP}, but there is no equivalence.

In browser \textsc{DH} or \textsc{DHE} indicates that \textsc{DH} in finite fields is used. 

\textsc{DH\textbf{E}}: ephemeral \textsc{DH}, i.\,e. use a new key for every connection or for each time interval.
Perfectly forward secrecy. Somebody taking over your system at time $t$ should not be able to decrypt any message prior to time $t$ (or $t-t_0$, where $t_0$ is the time for the ephemeral key).

\textsc{DH}: \textsc{Diffie-Hellman} with longterm keys.

Crypto parameters choose $p$ to have $\geq 2048$ bits and prime, work in $\F_p^{\ast}$, are okay with long-term use; ephemeral is to deal with stuff outside crypto.

\textsc{A(lice)} and \textsc{B(ob)} use share $g^{ab}$ after running it through a hash function to get a fixed-length string ($128$ or $256$ bits) with good distribution. Cryptographic hash functions are also:

\begin{itemize}
\item \emph{pre-image resistant},i.\,e. cannot find $g^{ab}$ from $h(g^{ab})$
\item \emph{second-pre-image resistant}, i.\,e. cannot find another pre-image given the first one
\item \emph{collision resistant}, i.\,e. cannot find two strings $m_1$ and $m_2$ with $h(m_1) = h(m_2)$.
\end{itemize}

\begin{remark}
We know that $m_1$ and $m_2$ exist, but we don't want to be able to compute them.
\end{remark}

In \textsc{DH} we use $h(g^{ab})$ as shared key in symmetric crypto, e.\,g. \textsc{AES}.

\section{ElGamal Encryption}\index{ElGamal Encryption}

General parameters: $g$, $n$

\textsc{Alice's} public key: $h_a = g^a$

\textsc{Alice's} secret key: $a$

Encryption: Pick random $0<k \leq n$, compute $r=g^k$, $c=h_a^k \cdot m$ (assume $m \in G$), with $m$ the message. We then send $(r,c)$.

Decryption: $\nicefrac{c}{r^a} = m'$. This works, i.\,e. $m=m'$ because
\[
	\frac{c}{r^a} = \frac{h_a^k \cdot m}{(g^k)^a} = \frac{\cancel{(g^a)^k} \cdot m}{\cancel{(g^a)^k}} = m
\]

In practice, \textsc{ElGamal} is not used like this because $m \not \in G$. Instead $c= AES_{h(h_a^k)}(m)$ and $m'$ is computed by first computing $K = h(r^a)$ and then $AES_K(c) = m'$. This corresponds to asymmetric \textsc{DH}.

\textsc{A} has longterm key, sender has $K$, $r$ as one time keys, but \textsc{DH} uses $h_a^K$ directly as key instead of transmitting $m$.

\subsection{ElGamal Signature Scheme}\index{ElGamal Signature Scheme}

Parameters as above; signature proves that the signer has access to $a$. Signature is on $h(m)$ not $m$ - fixed length; no algebraic relations.

\emph{Sign:} Pick \textbf{one-time} $0<k<n$ nonce (= number used only once), compute $r=g^k$, compute $s=k^{-1}(h(m) - a \cdot r) \bmod n$. The signature is $(r,s)$.

\emph{Verify:} Does $g^{h(m)}$ equal $h_a^r \cdot r^s$? A valid signature passes verification because:
\[
	h_a^r \cdot r^s = g^{ar} \cdot g^{k(k^{-1}(h(m) - ar))} = g^{h(m)} \qquad \surd
\]
Anybody knowing $a$ can sign. This becomes a problem if the $a$ being used, becomes known:

We see $(r,s)$ on $m$, so we can compute
\[
	s \cdot k = h(m) - a \cdot r \,\, \overset{\substack{\text{since } \\ s, k, r, h(m) \\ \text{ are known}}}{\implies} \,\, a = \frac{h(m) - s \cdot k}{r} \bmod n
\]

So we can recover \textsc{A}'s longterm secret $a$ with just one leaked nonce $k$.
We can also recover $a$ if $k$ is reused (see the homework), we can detect this from seeing repeated $r$.

\begin{remark}
The problem of re-using $k$, made the first \textsc{Playstation} open for attacks
\end{remark}

This means that \textsc{ElGamal} signatures are somewhat fragile. We can work around this by choosing $k$ pseudorandomly, i.\,e. \textsc{A} has two secrets: $a$ and $k_a$, compute $k$ as $k = h(k_a,m)$. This gives the same $k$ for repeated $m$, but the attack on repeated $k$ needs different $m$'s. If you want a shorter secret have a master secret key $s_a$ and derive $a=h(s_a, \alpha)$ - with $\alpha$ a string ``nonce key''.

Note that biases in $k$ also lead to breaks via the hidden number problem (we may take about this later), so would need really good randomness for each $k$ - or this construction.

Properties of hash functions are important here because a second pre-image of $h(m)$, i.\,e. a message $m'$ with $h(m) = h(m')$ can just use the same $(r,s)$.

If \textsc{E} has colliding messages $m_1$ and $m_2$, where $m_1$ is innocent, she can ask \textsc{A} to sign $m_1$ and use that $(r,s)$ as signature on $m_2$.

\section{Pohlig-Hellman Attack}\index{Pohlig-Hellman Attack}

For the \textsc{DDH} we know $g$, $g^a$, $g^b$, $g^c$, and the problem is, is $g^c = g^{ab}$?

Sometimes we can solve this without solving \textsc{CDH}:
\begin{align*}
&\F_{19}^{\ast} = \langle 2 \rangle  & h_a = 7, \qquad h_b = 11 \\
&g^c = 13 \overset{?}{=} DH(7,11) &
\end{align*}

We can figure out if $a$ was even or odd. $\ord(2) = 18$, so $18$ is smallest power of $2$ giving $1$. \\

If $a = 2a'$, then $h_a^9 = 2^{a 9} = 2^{2a' 9} = 2^{18a'} = 1$

else $a = 2a'+1$, then $h_a^9 = 2^{9(2a'+1)} = 2^{9} = 18$

$7^9 = 1$ same for $h_b$: $11^9 = 1$, so $a$ and $b$ are both even, so $a \cdot b$ is even.

Try $13^9$ to see whether it can be $2^{ab}: 13^9 = 18 \implies$ this is not $g^{ab} \implies$ we solved \textsc{DDH}.

This way we can solve the \textsc{DDH} whenever parity of $c$ and $ab$ does not match. \\

To make statements about probability of solving  \textsc{DDH} we work with sequences of triples and try to distinguish $(g^a, g^b,g^{ab})$ from $(g^a, g^b,g^c)$ with random $c$.

Avoid this \textsc{DDH} weakness by picking $g$ to have odd order, so in $\F_19^{\ast}$ use the subgroup generated by $2^2 = 4$ of order $9$. This doesn't solve all our problems, we can check whether $g^{ab}$ and $g^c$ match as third powers: \\

$7^{\nicefrac{18}{3}} = 7^6 = 1$ shows that $a$ is a multiple of $3$, thus $a \cdot b \equiv 0 \bmod 3$. Try whether $13$ matches this, here $13^6 = 11$, so another proof that $(7,11,13)$ is not a valid \textsc{DDH} triple. \\

Can we find out more about $a$ and $c$? We know:
\[
\left. \begin{array}{l} a \equiv 0 \bmod 2\\
a \equiv 0 \bmod 3  \end{array}\right\} a \equiv 0 \bmod 6 
\]

we also  know $0<a< 18$, i.\,e. $a=6$ or $a=12$. We can check the two possibilities and get $2^6 = 7$, so $a=6$.

\begin{remark}
This is sort of a ``brute-force'' approach.
\end{remark}

We know $c \equiv 1 \bmod 2$ and $c \not\equiv 0 \bmod 3$. So $c = c_0 + 3 \cdot c_1$, so $2^c = 2^{c_0 + 3 c_1}$, thus $2^{c \cdot 6} = 2^{6(c_0 + 3 c_1)} = 2^{6c_0}$ this gave $13^6=11$, so determine $c_0$, compare $11$ with $2^6$ (i.\,e. $c_0=1$) and $2^{12}$ (i.\,e. $c_0 = 2$).
\[
\text{Here: } 2^6 = 7 \qquad 2^{12} = 11 \qquad \implies \text{ so } c_0 = 2, c = 2 + 3 \cdot c_1
\]

With \textsc{CRT} we know $c \equiv 1 \bmod 2$ $c \equiv 2 \bmod 3$ $\implies c \equiv 5 \bmod 6$, so it could be $5$, $11$ or $17$.

We could just try these $3$ but we want a systematic way of computing \textsc{DL} in groups of composite orders.

\begin{align*}
&2^c = 2^{2+3 c_1} \qquad \text{ ,so} \\
&2^{3c_1} = \nicefrac{13}{4} = 8 \qquad \text{ , get } c_1 \text{ by computing} \\
&2^{3c_1 } =2^{6c_1} = 8^2 = 7 \qquad \text{ and then comparing to} \\
&2^0 = 1, 2^6 = 7 \text{ and } 2^{12}=11 \implies c_1=1
\end{align*}

$\implies$ complete set:
\[
c \equiv 1 \bmod 2 \\
c \equiv 5 \bmod 9 \\
\rightarrow c = 5
\]

This \emph{\textsc{Pohlig-Hellman} attack} solves the \textsc{DLP} in all subgroups of prime order and then combines the results using \textsc{CRT}.

\begin{example}\ \\
$\F_{37}^{\ast} = \langle 2 \rangle$, find $a = \log_2 17$,

the group has subgroups of order $2,3,4,9$ (and others of non-prime order)

\[
	17^{18} = 36 \implies a \equiv 1 \bmod 2
\]
\[
	17^{12} = 26, \text{ so } a \not\equiv 0 \bmod 3
\]

compare with $2^{12} = 26$ and with $2^{2 \cdot 12} = 10 \rightarrow a \equiv 1 \bmod 3$.

To get $a \bmod 4$, compute $(\nicefrac{17}{2})^9 = 27^9 = 36 \Rightarrow a \equiv 1 + 1 \cdot 2 \equiv 3 \bmod 4$.

Same for $\bmod 9$: $(\nicefrac{17}{2})^4 = 27^4 = 10 \implies a =1+2 \cdot 3 \equiv 7 \bmod 9 \implies a \equiv 7 \bmod 36$.

\end{example}